\documentclass[11pt]{article}
\usepackage{pdfpages}
% \pagestyle{empty}

\setlength{\oddsidemargin}{-0.25 in}
\setlength{\evensidemargin}{-0.25 in}
\setlength{\topmargin}{-0.9 in}
\setlength{\textwidth}{7.0 in}
\setlength{\textheight}{9.0 in}
\setlength{\headsep}{0.75 in}
\setlength{\parindent}{0.3 in}
\setlength{\parskip}{0.1 in}
\usepackage{epsf}
\usepackage{pseudocode}
\usepackage{listings}
\usepackage{tikz}
\usepackage{graphicx, color}
% \usepackage{times}
% \usepackage{mathptm}

\def\O{\mathop{\smash{O}}\nolimits}
\def\o{\mathop{\smash{o}}\nolimits}
\newcommand{\e}{{\rm e}}
\newcommand{\R}{{\bf R}}
\newcommand{\Z}{{\bf Z}}

\begin{document}
	\noindent HUID: 90949705 & _____
	\vspace{.2in}
	The dynamic programming solution to the Number partition problem involves creating an $n$X$b$ table. Every $(i,j)$ entry of the table contains a set of all the permutations of signs for the first $i$ numbers to make residue $j$. Once the table is filled the optimal solution is found by looking for the smallest $k$ such that entry $(n,k)$ is not empty. The recursive function would be
	\[X(i,j) = 
\end{document}